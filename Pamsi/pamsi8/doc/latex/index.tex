\hyperlink{class_program}{Program} sluzacy do wyznaczania skutecznosci algorytmu za pomoca pomiaru czasu

Aplikacja jest przykladem realizacji programu sluzacego do sprawdzania zlozonosci obliczeniowej algorytmu. Wykonuje ona pomiar czasu wykonywania zadanego programu. Aplikacja jest zbudowana na obiektach, dzieki czemu mozliwe jest szybkie wstawienie nowego algorytmu w postaci klasy dziedziczacej czesc metod po klasie bazowej.\hypertarget{index_etykieta-ogolne-informacje}{}\section{Ogolne informacje}\label{index_etykieta-ogolne-informacje}
\hyperlink{class_program}{Program} zostal wyposazony w funkcje wczytywania, wypisywania i zapisywania danych do pliku. Aplikacja byla sprawdzana na systemie operacyjnym Windows. Do poprawnego dzialania programu potrzebny jest plik z danymi zapisany w postaci\+: ilosc danych, wlasciwe dane. Mozliwe jest rowniez otwieranie wielu plikow danych przy pomocy jednego pliku. Musi on zawierac dane w postaci\+: ilosc sciezek do plikow, sciezki do plikow. \hyperlink{class_program}{Program} posiada rowniez funkcje automatycznego tworzenia pliku wejsciowego. Sam Program\+\_\+przykladowy powstal z mysla pokazania ze zadane operacje sa wykonywane poprawnie.\hypertarget{index_etykieta-algortym}{}\section{Algorytm}\label{index_etykieta-algortym}
Algorytm programu realizuje obsluge grafu. Oprocz samej struktury zawiera on rowniez przeszukiwania, ktorych zlozonosc jest badana. Przeszukiwania grafu to breadth\+\_\+first\+\_\+search, depth\+\_\+first\+\_\+search oraz best\+\_\+first\+\_\+search. Jako rezultat ostatniego z nich zwracana jest znaleziono droga od wierzcholka poczatkowego do wierzcholka docelowego. W pliku konfiguracja dostepne jest ustawienie dotyczace tego algorytmu. Mozna zmienic algortym ze standardowego wyszukiwujacego do samego konca wedlug omawianego schematu na algortym pomijajacy ostatnia krawedz pomiedzy wierzcholkami (wystarczy ze odleglosc miedzy wierzcholkami bedzie wynosic 1 i algorytm zakonczy dzialanie). Wplywa to znacznie na otrzymywane wyniki. Algorytm tworzy rowniez plik o rozszerzeniu .tex, ktory zawiera badany graf. 