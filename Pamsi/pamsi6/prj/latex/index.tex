\hyperlink{class_program}{Program} sluzacy do wyznaczania skutecznosci algorytmu za pomoca pomiaru czasu

Aplikacja jest przykladem realizacji programu sluzacego do sprawdzania zlozonosci obliczeniowej algorytmu. Wykonuje ona pomiar czasu wykonywania zadanego programu. Aplikacja jest zbudowana na obiektach, dzieki czemu mozliwe jest szybkie wstawienie nowego algorytmu w postaci klasy dziedziczacej czesc metod po klasie bazowej.\hypertarget{index_etykieta-ogolne-informacje}{}\section{Ogolne informacje}\label{index_etykieta-ogolne-informacje}
\hyperlink{class_program}{Program} zostal wyposazony w funkcje wczytywania, wypisywania i zapisywania danych do pliku. Aplikacja byla sprawdzana na systemie operacyjnym Windows. Do poprawnego dzialania programu potrzebny jest plik z danymi zapisany w postaci\-: ilosc danych, wlasciwe dane. Mozliwe jest rowniez otwieranie wielu plikow danych przy pomocy jednego pliku. Musi on zawierac dane w postaci\-: ilosc sciezek do plikow, sciezki do plikow. Sam Program\-\_\-przykladowy powstal z mysla pokazania ze zadane operacje sa wykonywane poprawnie.\hypertarget{index_etykieta-algortym}{}\section{Algorytm}\label{index_etykieta-algortym}
Algorytm zawarty w progeramie realizuje zadania na drzewie, do ktorego moga byc wpisywane pary danych wartosc-\/klucz. \hyperlink{class_drzewo}{Drzewo} jest sortowane podczas wpisywania kolejnych danych, co ulatwia i przyspiesza pozniejszy odczyt. Na potrzeby cwiczenia zademonstrowano dzialanie podstawowych metod klasy odpowiedzialnej za dzialania na drzewie. 